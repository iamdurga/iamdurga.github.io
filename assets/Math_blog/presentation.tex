\documentclass[12pt]{beamer}
\usepackage{tikz}
\usetikzlibrary{automata}
\usetheme{EastLansing}
\title{Graphs and automata in Ti\textit{k}Z}
\title{PRESENTATION SLIDE WITH BEAMER.}
\author{\tiny Durga pokharel}
\institute{Central Department of mathematics}
\date{\today}
\begin{document}
\begin{frame}
\maketitle

\end{frame}
\begin{frame}{ HOW TO MAKE NICE PRESENTATIONON BEAMER.}
Hello,this is me Durga pokharel from palpa.I am first semister student with roll.no.43.

I want to give huge thanks to the respected professor to give me this opertinuty to give the presentation in the topic digraph.
\end{frame}
\begin{frame}[t]{contents}
  
 \begin{enumerate}
    \item  Graphs
    \item  Digraphs
    \item Arcs
    \item vertex
    \item Loop
    \item In-degree,out-degree
    \item Path,directed path,simplepath
    \item cycle
    \item connected graph
    \item partial digraph
    \item sub-digraph
 \end{enumerate}


\end{frame}


  

\begin{frame}[t]{Graph}
A graph is a set G = (V,A) where V is the set of vertices and A is the set of arcs.\\
     OR
We have define the graph as a set and a set of relation on that set .
\begin{tikzpicture}
\node[circle,draw = black](v1)at(0,0){A};
\node[circle,draw = black](v2) at (3,4){B};
\node[circle,draw = black](v3) at (3,0){C};
\draw[->](v1)--(v2);
\draw[->](v2)--(v3);
\draw[->](v3)--(v1);
\end{tikzpicture}\\
Where A = \{(A,B),(B,C),(C,A)\}\\
      V = \{A,B,C\}

\end{frame}

\end{document}